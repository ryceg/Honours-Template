% chapter1.tex (Chapter 1 of the thesis)

\renewcommand*{\thefootnote}{\arabic{footnote}}
\setcounter{footnote}{0}

\chapter{Hello, world!}

This first chapter is typically where one would see a literature review and methodology, at the start of the exegesis.
Rather than bog the document down with irrelevant example material, this will be instructive text that can be used as a reference.
Please note that it is \emph{not} a tutorial, and is just example code that you can copy.
When you're ready to write your \emph{own} thesis, you can comment out these lines by highlighting them, and pressing \lstinline{ctrl} (or \lstinline{cmd} on a Mac) + \lstinline{/} to comment it out.

\section{LaTex}

\begin{quotation}
    `[\ldots] LaTex is like the old printing presses, except it's on computer screens, so it's a lot faster.'
\end{quotation}
LaTex was created in 1985, and is typesetting software designed to create beautiful documents. 
It is best known for its ability to handle complex mathematical equations, and do just about everything under the sun.
It is \emph{not} word processing software; it shares similar features to Word, but is notably different in that Word is a what-you-see-is-what-you-get (WYSIWYG) editor, whereas LaTex separates writing from the formatting.
This means in practice that you deal with stuff that looks like \emph{code} in LaTex, and compile it into the finished product. 

\subsection{Why use LaTex?}
Many articles have been written about why you should use LaTex, and a few choice have been supplied for reference.
Ultimately, LaTex produces beautiful documents, but whether it is worth the stress of learning a new system is up to the reader.
This document has been created as an instructional tutorial as a starting point.

\subsubsection{The Author's Reasons}
My personal reasons were as follows:
\begin{itemize}
    \item Stability: LaTex is a relic of the 1980s, but this works to its advantage when handling large documents, which are broken up into bite-sized chunks. This system means that you can shift sections around without any messy copy--pastes.
    \item Flexibility: I used LaTex for my Honours exegesis. It is able to not only insert PDFs, but even create uniform cover pages for my compositions.
    \item Variables: I can define a variable that might change and use that, obviating any find-and-replace-all issues.
    \item Integration: I am able to use Zotero to save a source to the \lstinline{works-cited.bib} file, \lstinline{\autocite} to automatically cite a source, and back it up to GitHub.
    \item Version control: A save button, combined with Time Machine- using GitHub to save my exegesis to the cloud, I can track \textbf{every} change made to the document. Version control works with plain text files (i.e. LaTex), but not proprietary formats such as .docx
    \item Comments: I can comment out sections, preserving it without it showing up on the finished document. This means that I can implement TODOs, make notes, and have people make changes.
\end{itemize}


\newpage
\section{New Section}
Here we have started a new page to show how the headers work. The
text in the header should be the last section title declared at
the end of the current page.

This new paragraph shows how to set\index{index items}index items
and\index{index items!subindex items} subindex items.

\subsection{New Subsection}
Here's a subsection with some simple maths $a^2+b^2=c^2$.

\subsubsection{subsubsection}
Here's a\index{subsubsection}subsubsection\ldots oooooooohh wow
wee!!!!!!

\newpage
Some more text to check indent and show how references work

