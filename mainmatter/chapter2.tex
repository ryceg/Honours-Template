% chapter2.tex (Chapter 2 of the thesis)

\chapter{Setup and Config}\label{ch:chapter2}

Rename this to your preferred chapter title.

\section{Setup}

In order to use this template, you will need to set up a method of compiling it. 
The simplest option is to use https://overleaf.com, an online LaTex editor.
This is the fastest way to get it up and running, but comes with the disappointing disadvantage of locking Zotero integration behind a premium paywall.
Don't bother trying to do references by hand. 
Pay the money, or go for my preferred option of using VSCode.

% There are many other options; see here https://tex.stackexchange.com/questions/339/latex-editors-ides/390058#390058 

\subsection{VSCode}
VSCode (https://code.visualstudio.com/) is a source code editor, and includes Git version control. 
In order for it to be able to understand LaTex, we must first give it the required libraries; install TexLive at https://www.tug.org/texlive/acquire-netinstall.html 
You can untick the box for `frontend'- since we're going to be writing in VSCode, we don't need it.

Download VSCode, install it, and boot it up.
Then, grab the following:

https://marketplace.visualstudio.com/items?itemName=James-Yu.latex-workshop

I would also recommend:

https://marketplace.visualstudio.com/items?itemName=lkytal.pomodoro

https://marketplace.visualstudio.com/items?itemName=Gruntfuggly.todo-tree

\subsection{Zotero}

For Zotero integration, install Zotero: https://www.zotero.org/

Then, install the BetterBibTex extension in Zotero: https://retorque.re/zotero-better-bibtex/

Then, finally, install the Zotero LaTeX extension in VSCode: https://marketplace.visualstudio.com/items?itemName=bnavetta.zoterolatex

\subsection{GitHub}

For GitHub integration, install GitHub Desktop\footnote{While not necessary, setting up a new GitHub repository is easiest through the user interface.}: https://desktop.github.com/ 

Create an account on GitHub, and optionally register as a student for free private repositories here\footnote{There's also some other great stuff in there, specifically the free domain registry, PomoDone, and pro TypeForm. Seriously, check it out.}: https://education.github.com/pack 

Navigate to this 